\documentclass[10pt,a4paper,titlepage]{article}
\usepackage[utf8]{inputenc}
\usepackage{polski}
\usepackage{amsmath}
\usepackage{amsfonts}
\usepackage{amssymb}
\usepackage{graphicx}
\usepackage{float}
\usepackage{indentfirst}
\author{Adrian Frydmański, Dawid Gracek}
\title{Systemy Inteligentnego Przetwarzania\\--- Projekt ---\\Wahadło odwrócone (sieć neuronowa)}
\begin{document}
	\maketitle
	\clearpage
	\section{Wstęp teoretyczny}
	Odwrócone wahadło niewiele różni się od swojego ,,zwykłego'' odpowiednika. Jest swobodnie wiszącym prętem przymocowanym do wózka. Wózek z kolei ma możliwość poruszania się wzdłuż osi (w jednym wymiarze, acz nic nie stoi na przeszkodzie, żeby rozszerzyć problem wahadła do dwóch wymiarów). Układ ten:
	\begin{itemize}
		\item posiada dwa punkty równowagi: stabilny, kiedy wahadło spoczywa w położeniu dolnym i niestabilny, kiedy wahadło skierowane jest pionowo ku górze,
		\item jest tzw. obiektem niedosterowanym, ponieważ wielkości sterowanych możemy wyróżnić więcej niż jest wejść w układzie.
	\end{itemize}

	\begin{figure}[H]
		\center
		\includegraphics[width=.6\textwidth]{w.png}
		\caption{Model wahadła}
	\end{figure}
	
	Jedyną wielkością, która wpływa na stan układu jest siła przyłożona do wózka, którego przemieszczanie się wprawia w ruch wahadło. Taki układ regulacji może mieć kilka celów:
	\begin{itemize}
		\item stabilizacja wahadła w położeniu górnym,
		\item regulacja położenia wózka w odniesieniu do całego stanowiska,
		\item realizacja algorytmów umożliwiających rozbujanie wahadła z pozycji dolnej i doprowadzenie go to pozycji górnej.
	\end{itemize}
	
	Ów projekt zakłada realizację 2. pierwszych celów poprzez generowanie odpowiednich nastaw regulatora kontrolującego ruch wózka i wprowadzenie go do położenia górnego ze stanu początkowego, gdy wahadło jest nachylone względem ziemi pod kątem mniejszym, niż $90^\circ$. Nastawy mają być generowane przez wyuczoną sieć neuronową.
	
	Matematyczny model wahadła jest opisany na stronie jtjt.pl\footnote{http://jtjt.pl/www/pages/odwrocone-wahadlo/LMIP.pdf} i stamtąd właśnie zaczerpnięte są obliczenia, z których korzysta symulacja.
	
	\section{Implementacja}
	Środowisko, w jakim została przeprowadzona symulacja i tworzenie sieci neuronowej, to Matlab R2017a. Dane testowe do uczenia sieci zostały pozyskane przez funkcję\footnote{Model i funkcja dostępne na https://github.com/Jarczyslaw/Inverted-Pendulum} do strojenia regulatora liniowo kwadratowego w zależności od ,,parametrów środowiska'':
	\begin{itemize}
		\item masa wózka,
		\item masa wahadła,
		\item długość od mocowania do środka ciężkości wahadła,
		\item współczynnik tarcia wózka.
	\end{itemize}

	W głównej funkcji, \texttt{lqr\_training}, zostało wygenerowane 100.000 losowych zestawów danych i znaleziono dla każdego z nich odpowiednie nastawy regulatora.\footnote{Cały kod dostępny w repozytorium https://github.com/Adrian94F/SIP} Następnie w narzędziu \texttt{nntool} została wygenerowana sieć neuronowa, która na owych zestawach nauczyła się dobierać parametry regulatora. Sprawdzono, że dla 4 neuronów --- tylu, co wyjść --- sieć działa wystarczająco dobrze. Sieć posiada jedną warstwę ukrytą. Przyjęto następujące wielkości podzbiorów zbioru 100.000 zestawów testowych:
	\begin{itemize}
		\item 70\% --- zbiór uczący
		\item 15\% --- zbiór walidacyjny
		\item 15\% --- zbiór testowy
	\end{itemize}
	
	\begin{figure}[H]
		\center
		\includegraphics[width=\textwidth]{nn.png}
		\caption{Wygenerowana sieć i proces uczenia}
	\end{figure}

	\begin{figure}[H]
		\center
		\includegraphics[width=.88\textwidth]{perf.png}
		\caption{Wykres błędów od epok treningowych}
	\end{figure}

	\begin{figure}[H]
		\center
		\includegraphics[width=.88\textwidth]{errhist.png}
		\caption{Histogram błędów}
	\end{figure}
	
	Po wygenerowaniu sieci został ponownie wylosowany zbiór zmiennych środowiska, na podstawie których sieć znalazła odpowiednie nastawy regulatorów. Została uruchomiona symulacja i przebiegła ona pomyślnie.
	
	\begin{figure}[H]
		\center
		\includegraphics[width=\textwidth]{wyniksymulacji1.png}
		\caption{Osiągnięty stan równowagi podczas symulacji}
	\end{figure}

	\begin{figure}[H]
		\center
		\includegraphics[width=\textwidth]{wyniksymulacji2.png}
		\caption{Wyniki symulacji (od góry): kąt i pochodna kąta pomiędzy wahadłem, a pionem, prędkość i pozycja wózka, przyłożona siła do wózka i zadana wartość zadana}
	\end{figure}
	
	\section{Podsumowanie}
	Projekt pokazał, że sieć neuronową można zaprząc do przeróżnych zadań, nawet do dostosowywania nastaw regulatora w wahadle odwróconym. Jest to możliwe do zrobienia i proste w implementacji w takim środowisku, jak Matlab, aczkolwiek naszym zdaniem zbędne, gdyż funkcja strojąca regulator, dzięki której uzyskaliśmy dane do nauki sieci, działa wystarczająco szybko, a wynik działania sieci neuronowej zawsze będzie obarczony pewnym błędem w stosunku do wyniku działania owej funkcji.
	
\end{document}