\documentclass[10pt,a4paper,titlepage]{article}
\usepackage[utf8]{inputenc}
\usepackage{polski}
\usepackage{amsmath}
\usepackage{amsfonts}
\usepackage{amssymb}
\usepackage{graphicx}
\usepackage{indentfirst}
\author{Adrian Frydmański, Dawid Gracek}
\title{Systemy Inteligentnego Przetwarzania\\--- Projekt ---\\Wahadło odwrócone (sieć neuronowa)}
\begin{document}
	\maketitle
	\clearpage
	\section{Wstęp teoretyczny}
	Odwrócone wahadło niewiele różni się od swojego ,,zwykłego'' odpowiednika. Jest swobodnie wiszącym prętem przymocowanym do wózka. Wózek z kolei ma możliwość poruszania się wzdłuż osi (w jednym wymiarze, acz nic nie stoi na przeszkodzie, żeby rozszerzyć problem wahadła do dwóch wymiarów). Układ ten:
	\begin{itemize}
		\item posiada dwa punkty równowagi: stabilny, kiedy wahadło spoczywa w położeniu dolnym i niestabilny, kiedy wahadło skierowane jest pionowo ku górze,
		\item jest tzw. obiektem niedosterowanym ponieważ wielkości sterowanych możemy wyróżnić więcej niż jest wejść w układzie.
	\end{itemize}
	
	Jedyną wielkością, która wpływa na stan układu jest siła przyłożona do wózka, którego przemieszczanie się wprawia w ruch wahadło. Taki układ regulacji może mieć kilka celów:
	\begin{itemize}
		\item stabilizacja wahadła w położeniu górnym,
		\item regulacja położenia wózka w odniesieniu do całego stanowiska,
		\item realizacja algorytmów umożliwiających rozbujanie wahadła z pozycji dolnej i doprowadzenie go to pozycji górnej.
	\end{itemize}
	
	Ów projekt zakłada realizację 2. pierwszych celów poprzez generowanie odpowiednich nastaw regulatora kontrolującego ruch wózka i wprowadzenie go do położenia górnego ze stanu początkowego, gdy wahadło jest nachylone względem ziemi pod kątem mniejszym, niż $90^\circ$.
	
	\section{Implementacja}	
	Opisać: co zrobiliśmy? + jak zrobiliśmy? + jak uzyskaliśmy dane? + topologia sieci + proces uczenia sieci
	
	\section{Podsumowanie}
	prezentacja wniosków + podsumowanie, że dziaua ;)
\end{document}